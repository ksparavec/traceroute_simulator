\documentclass[10pt,a4paper]{article}
\usepackage[utf8]{inputenc}
\usepackage[ngerman]{babel}
\usepackage[margin=2cm]{geometry}
\usepackage{graphicx}
\usepackage{tikz}
% \usepackage{enumitem}  % Not available
% \usepackage{fancyhdr}  % Not available
\usepackage{xcolor}
\usepackage{listings}
\usepackage{hyperref}
% \usepackage{courier}  % Not available

% Color definitions
\definecolor{darkblue}{RGB}{0,51,102}
\definecolor{lightgray}{RGB}{240,240,240}
\definecolor{darkgreen}{RGB}{0,128,0}

% Page style - commented out as fancyhdr is not available
% \pagestyle{fancy}
% \fancyhf{}
% \fancyhead[L]{\textbf{SSH Restricted Access} - Sicherheitsimplementierung}
% \fancyfoot[C]{\thepage}
% \renewcommand{\headrulewidth}{0.4pt}

% Code listing style
\lstset{
    basicstyle=\ttfamily\scriptsize,
    backgroundcolor=\color{lightgray},
    breaklines=true,
    frame=single,
    frameround=tttt,
    numbers=left,
    numberstyle=\tiny\color{gray},
    keywordstyle=\color{darkblue}\bfseries,
    commentstyle=\color{darkgreen},
    stringstyle=\color{red}
}

\title{\textbf{SSH Restricted Access}\\
\large Sicherheitsanalyse und Implementierung\\
\normalsize Traceroute-Simulator Projekt}
\author{Ansible SSH-Restricted-Access Lösung}
\date{\today}

\begin{document}

% Page 1: Übersicht und OS-Sicherheit
\maketitle
% \thispagestyle{fancy}  % Commented out as fancyhdr not available

\section*{1. Höchste OS-Sicherheitsstufe mit authorized\_keys}

\subsection*{OpenSSH 7.2+ \texttt{restrict} Option}
\begin{itemize}
    \item \textbf{Modernste Sicherheit}: Nutzt \texttt{restrict} Option (ab OpenSSH 7.2)
    \item \textbf{Vollständige Einschränkungen}: Deaktiviert ALLE gefährlichen SSH-Features
    \item \textbf{Defense-in-Depth}: Zusätzliche explizite Einschränkungen trotz \texttt{restrict}
\end{itemize}

\begin{lstlisting}[language=bash,title=authorized\_keys Konfiguration (NUR Ansible Controller erlaubt!)]
restrict,no-port-forwarding,no-X11-forwarding,no-agent-forwarding,
no-pty,from="10.1.2.3",command="/usr/local/bin/tracersh" <key>
\end{lstlisting}

\subsection*{Mehrschichtige Sicherheitsarchitektur}
\begin{itemize}    \item \textbf{Netzwerk-Ebene}: \texttt{from="10.1.2.3"} - \textcolor{red}{\textbf{NUR der Ansible Controller darf verbinden!}}
    \item \textbf{Authentifizierung}: Nur Public-Key, kein Passwort
    \item \textbf{Befehlsebene}: Erzwungene Ausführung von \texttt{tracersh}
    \item \textbf{Shell-Ebene}: Benutzerdefinierte Shell verhindert interaktiven Zugriff
\end{itemize}

\subsection*{Dateisystem-Sicherheit}
\begin{center}
\begin{tabular}{|l|c|c|l|}
\hline
\textbf{Datei/Verzeichnis} & \textbf{Besitzer:Gruppe} & \textbf{Rechte} & \textbf{Sicherheitsvorteil} \\
\hline
\texttt{/home/traceuser} & root:tracegroup & 750 & Keine Benutzeränderung \\
\texttt{\textasciitilde/.ssh} & root:tracegroup & 750 & Geschützte SSH-Konfig \\
\texttt{authorized\_keys} & root:tracegroup & 640 & Nur lesbar via Gruppe \\
\texttt{/usr/local/bin/tracersh} & root:root & 755 & Systemweit, unveränderlich \\
\hline
\end{tabular}
\end{center}

\textbf{Sicherheitsprinzipien:}
\begin{itemize}    \item Root-Besitz verhindert Privilege Escalation
    \item Gruppenbasierte Zugriffskontrolle
    \item Principle of Least Privilege durchgängig angewendet
\end{itemize}

\newpage

% Page 2: Bash Script Sicherheit
\section*{2. Moderne Bash-Script Sicherheitsanforderungen}

\subsection*{Implementierte Sicherheitsmaßnahmen}

\begin{lstlisting}[language=bash,title=Härtungsblock (MUSS zuerst kommen)]
# ---- HARDENING BLOCK (must be first) ----
umask 077                              # Restriktive Dateierstellung
export LC_ALL=C LANG=C                 # Vorhersehbare Locale
IFS=$'\n\t'                           # Sicherer Feldtrenner
PATH='/usr/sbin:/usr/bin:/sbin:/bin'  # Fester PATH
unset BASH_ENV ENV CDPATH GLOBIGNORE  # Gefaehrliche Vars entfernen
unalias -a 2>/dev/null || true        # Alle Aliase entfernen
# ----------------------------------------
set -euo pipefail                      # Strict Mode
\end{lstlisting}

\subsection*{Vollständige Sicherheitsanforderungen}

\begin{center}
\begin{tabular}{|l|l|c|}
\hline
\textbf{Anforderung} & \textbf{Implementierung} & \textbf{Status} \\
\hline
Umgebungssäuberung & \texttt{LC\_ALL=C LANG=C} & OK \\
Sicheres IFS & \texttt{IFS=\$'\\n\\t'} & OK \\
Fester PATH & \texttt{PATH='/usr/sbin:/usr/bin:/sbin:/bin'} & OK \\
Variablen-Bereinigung & \texttt{unset BASH\_ENV ENV CDPATH} & OK \\
Alias-Entfernung & \texttt{unalias -a} & OK \\
Strict Mode & \texttt{set -euo pipefail} & OK \\
Input-Validierung & Regex \texttt{\^{}10\textbackslash.\textbackslash d+\textbackslash.\textbackslash d+\textbackslash.\textbackslash d+\$} & OK \\
Command Injection Schutz & \texttt{awk '\{print \$1\}'} Extraktion & OK \\
Timeout-Schutz & \texttt{timeout --kill-after=5s 60s} & OK \\
Signal-Behandlung & \texttt{trap} für HUP, INT, TERM & OK \\
Read-only Variablen & \texttt{readonly} Deklarationen & OK \\
Fehlerbehandlung & Dedizierte \texttt{error\_exit} Funktion & OK \\
\hline
\end{tabular}
\end{center}

\subsection*{Zugriffskontroll-Logik}

\begin{enumerate}[leftmargin=*]
    \item \textbf{Keine SSH-Verbindung}: „Nur SSH-Zugriff erlaubt"
    \item \textbf{SSH mit TTY ohne Befehl}: „Nur Traceroute-Ausführung erlaubt"
    \item \textbf{SSH ohne Original-Befehl}: „Nur Traceroute-Ausführung erlaubt"
    \item \textbf{Ungültige IP-Adresse}: „\# input invalid: <input> - must be valid 10.x.x.x IPv4 address"
\end{enumerate}

\subsection*{Input-Sanitierung}
\begin{itemize}    \item Nur erstes Wort des SSH-Befehls wird verarbeitet
    \item Strikte Regex-Validierung für 10.x.x.x Adressen
    \item Alle weiteren Parameter werden ignoriert
    \item Keine Shell-Expansion oder Variablen-Substitution
\end{itemize}

\newpage

% Page 3: Dokumentation und Wartbarkeit
\section*{3. Dokumentation und Erweiterbarkeit}

\subsection*{Umfassende Dokumentationsstruktur}

\begin{itemize}    \item \textbf{README.md}: 430+ Zeilen Benutzerdokumentation
    \item \textbf{SECURITY-ANALYSIS.md}: Detaillierte Sicherheitsanalyse
    \item \textbf{Inline-Kommentare}: Jede Funktion dokumentiert
    \item \textbf{Jinja2-Header}: Deployment-Kontext in Templates
    \item \textbf{YAML-Kommentare}: Konfigurationsbeispiele mit Erklärungen
\end{itemize}

\subsection*{Code-Organisation für einfache Wartung}

\begin{lstlisting}[title=Verzeichnisstruktur]
ssh-restricted-access/
|- config/default.yml         # Zentrale Konfiguration
|- deploy.yml                 # Ein-Befehl-Deployment
|- remove.yml                 # Saubere Entfernung
|- test.yml                   # Automatisierte Tests
|- roles/ssh_restricted_access/
   |- defaults/main.yml       # Dokumentierte Defaults
   |- tasks/                  # Modulare Aufgaben
   |- templates/              # Anpassbare Templates
\end{lstlisting}

\subsection*{Erweiterungspunkte}

\begin{itemize}    \item \textbf{Tool-Unterstützung}: Einfaches Hinzufügen neuer Netzwerk-Tools
    \item \textbf{Validierungsmuster}: Anpassbare Regex für andere IP-Bereiche  
    \item \textbf{Logging}: Optional mit Rotation und SIEM-Integration
    \item \textbf{Zentrale Benutzerverwaltung}: FreeIPA/LDAP-Support integriert
\end{itemize}

\subsection*{Testabdeckung}

\begin{center}
\begin{tabular}{|l|l|}
\hline
\textbf{Testbereich} & \textbf{Abdeckung} \\
\hline
Konnektivität & SSH-Verbindung, Schlüssel-Auth \\
Befehlsausführung & Gültige/ungültige Targets \\
Sicherheitsrestriktionen & Verbotene Befehle, Shell-Zugriff \\
Fehlerbehandlung & Timeouts, fehlende Tools \\
Idempotenz & Mehrfache Ausführungen \\
\hline
\end{tabular}
\end{center}

\textbf{Vorteile für verschiedene Nutzergruppen:}
\begin{itemize}    \item \textbf{Entwickler}: Klare Struktur, Template-basiert, erweiterbar
    \item \textbf{Administratoren}: YAML-Konfiguration, keine Programmierung nötig
    \item \textbf{Security-Teams}: Vollständige Dokumentation, Audit-Logs
    \item \textbf{Management}: Compliance-ready, wartbar, skalierbar
\end{itemize}

\newpage

% Page 4: Wartbarkeit für Administratoren und Architektur
\section*{4. Einfache Wartung für Systemadministratoren}

\subsection*{Keine Programmierkenntnisse erforderlich}

\textbf{Einfache Befehle für tägliche Aufgaben:}
\begin{lstlisting}[language=bash]
# Deployment mit Inline-Inventory (empfohlen)
ansible-playbook -i "192.168.122.230," deploy.yml

# Test der Installation (Quick-Test)
ansible-playbook -i "host," test.yml -e "run_quick_test=true"

# Test der Installation (Vollständig)
ansible-playbook -i "host," test.yml

# Saubere Entfernung
ansible-playbook -i "host," remove.yml

# Mit Umgebungsvariablen (keine Config-Änderung nötig)
export TRACEROUTE_SIMULATOR_TRACEUSER_PKEY_FILE="/tmp/id_traceuser.pub"
export TRACEROUTE_SIMULATOR_FROM_HOSTS="10.1.2.3"  # NUR Ansible Controller!
ansible-playbook -i "host," deploy.yml
\end{lstlisting}

\subsection*{YAML-basierte Konfiguration}
\begin{itemize}    \item Alle Einstellungen in \texttt{config/default.yml}
    \item Klare Variablennamen mit Beschreibungen
    \item Beispielwerte für alle Parameter
    \item Keine Code-Änderungen notwendig
\end{itemize}

\subsection*{Sicherheitsarchitektur-Diagramm}

\begin{center}
\tikzstyle{block} = [rectangle, draw, fill=blue!10, text width=10cm, text centered, rounded corners, minimum height=1.2em, font=\small]
\tikzstyle{check} = [rectangle, draw, fill=green!10, text width=9cm, text centered, font=\footnotesize]
\tikzstyle{arrow} = [thick,->,>=stealth]

\begin{tikzpicture}[node distance=0.7cm]
\node [block] (client) {\textbf{SSH Client: Ansible Controller (10.1.2.3)}\\ssh traceuser@router 10.8.8.8\\\textcolor{red}{\textbf{STRIKTE ANFORDERUNG: NUR diese IP erlaubt!}}};
\node [block, below=of client] (openssh) {\textbf{OpenSSH Server} (Zielknoten/Router)};
\node [check, below=0.3cm of openssh] (check1) {1. Netzwerk-Check: from="10.1.2.3" (EXAKTE IP-Prüfung!)\\2. Schlüssel-Auth: authorized\_keys\\3. Restriktionen: restrict, no-pty, no-*-forwarding\\4. Erzwungener Befehl: command="/usr/local/bin/tracersh"};
\node [block, below=1.3cm of check1] (tracersh) {\textbf{tracersh (Restricted Shell)}};
\node [check, below=0.3cm of tracersh] (check2) {1. Umgebungs-Härtung\\2. SSH-Kontext-Validierung\\3. Input-Extraktion (nur erstes Wort)\\4. Regex-Validierung (nur 10.x.x.x)\\5. Tool-Auswahl und Timeout-Ausführung};
\node [block, below=1.3cm of check2] (tools) {\textbf{System-Tools} (traceroute/mtr)\\ICMP-Modus, keine DNS-Auflösung, CSV-Output};

\draw [arrow] (client) -- (openssh);
\draw [arrow] (openssh) -- (check1);
\draw [arrow] (check1) -- (tracersh);
\draw [arrow] (tracersh) -- (check2);
\draw [arrow] (check2) -- (tools);
\end{tikzpicture}
\end{center}

\subsection*{Berechtigungsanforderungen}
\begin{center}
\begin{tabular}{|l|l|l|}
\hline
\textbf{Aufgabe} & \textbf{Ansible-Playbook} & \textbf{Traceroute-Ausführung} \\
\hline
Benutzer & Admin mit sudo & Eingeschränkter Benutzer \\
Rechte & Volle sudo (root) & \textbf{KEINE Admin-Rechte} \\
SSH-Zugriff & Passwort oder Schlüssel & Nur Schlüssel (forced cmd) \\
Zweck & Deployment/Config & Nur traceroute/mtr \\
\hline
\end{tabular}
\end{center}

\textcolor{red}{\textbf{STRIKTE ANFORDERUNG:}} Der SSH-Client ist AUSSCHLIESSLICH der Ansible Controller (10.1.2.3). Keine anderen IPs sind erlaubt!

\subsection*{Wartungsaufgaben}
\begin{center}
\begin{tabular}{|l|l|l|}
\hline
\textbf{Frequenz} & \textbf{Aufgabe} & \textbf{Befehl/Aktion} \\
\hline
Täglich & Log-Prüfung & \texttt{tail -f /var/log/tracersh.log} \\
Wöchentlich & Verbindungstests & \texttt{ansible-playbook -i "host," test.yml} \\
Monatlich & Berechtigungen prüfen & \texttt{ls -la /home/traceuser/.ssh} \\
Quartal & Schlüssel-Rotation & Config anpassen, deploy.yml \\
Jährlich & Sicherheitsaudit & Externe Prüfung \\
\hline
\end{tabular}
\end{center}

\textbf{Fazit:} Die Lösung erfüllt alle Anforderungen mit höchster Sicherheit bei gleichzeitiger Wartbarkeit durch Administratoren ohne Programmierkenntnisse.

\end{document}