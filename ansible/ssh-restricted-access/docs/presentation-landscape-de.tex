\documentclass[11pt,landscape]{article}
\usepackage[a4paper,landscape,margin=2cm]{geometry}
\usepackage[T1]{fontenc}
\usepackage[utf8]{inputenc}
\usepackage[ngerman]{babel}
\usepackage{graphicx}
\usepackage{xcolor}
\usepackage{listings}
\usepackage{hyperref}
\usepackage{array}
\usepackage{tabularx}
\usepackage{booktabs}
\usepackage{fancyhdr}
\usepackage{courier}
\usepackage{helvet}
\renewcommand{\familydefault}{\sfdefault}

\definecolor{darkblue}{RGB}{0,51,102}
\definecolor{lightgray}{RGB}{240,240,240}
\definecolor{darkgreen}{RGB}{0,128,0}

\pagestyle{fancy}
\fancyhf{}
\fancyfoot[L]{\footnotesize Kresimir Sparavec, ZIT BB D4.3 externe Unterstützungskraft}
\fancyfoot[R]{\footnotesize \thepage}
\renewcommand{\headrulewidth}{0pt}
\renewcommand{\footrulewidth}{0.4pt}

\lstset{
    basicstyle=\ttfamily\small,
    backgroundcolor=\color{lightgray},
    breaklines=true,
    frame=single,
    numbers=left,
    numberstyle=\tiny,
    keywordstyle=\bfseries,
    commentstyle=\itshape,
    stringstyle=\ttfamily
}

\setlength{\parindent}{0pt}
\setlength{\parskip}{0.3cm}

% Increase table font sizes
\renewcommand{\arraystretch}{1.2}

\begin{document}

% Seite 1
\section*{\textbf{\huge SSH Restricted Access - Sicherheitsimplementierung}}
\vspace{-0.3cm}
\noindent\rule{\textwidth}{0.4pt}
\vspace{0.3cm}
\subsection*{\Large H\"ochste Sicherheit f\"ur Traceroute-Ausf\"uhrung}

\textbf{\large Kernmerkmale der L\"osung:}
\begin{itemize}
    \item \textbf{Strikte Zugriffskontrolle:} NUR Ansible Controller (10.1.2.3) erlaubt
    \item \textbf{OpenSSH 7.2+ restrict:} Modernste Sicherheitsmechanismen  
    \item \textbf{Keine Admin-Rechte:} Traceroute l\"auft als eingeschr\"ankter Benutzer
    \item \textbf{Validierung:} Nur 10.x.x.x Ziele erlaubt, CSV-Output
    \item \textbf{Defense-in-Depth:} Mehrschichtige Sicherheitsarchitektur
\end{itemize}

\textbf{\large Berechtigungsmatrix:}

\begin{center}
{\large
\begin{tabular}{|l|l|l|}
\hline
\textbf{Aufgabe} & \textbf{Ansible-Playbook Ausf\"uhrung} & \textbf{Traceroute-Ausf\"uhrung} \\
\hline
Benutzer & Admin mit sudo-Zugriff & traceuser (eingeschr\"ankt) \\
Privilegien & Volle sudo (root) & \textbf{KEINE Admin-Rechte!} \\
SSH-Zugriff & Beliebige IP, Passwort/Key & \textbf{NUR 10.1.2.3}, nur Key + forced cmd \\
Zweck & Deployment/Konfiguration & Ausschlie\ss{}lich traceroute/mtr \\
\hline
\end{tabular}
}
\end{center}

\textbf{\large Sicherheitsprinzipien:}

\begin{center}
{\large
\begin{tabular}{|l|l|}
\hline
\textbf{Prinzip} & \textbf{Umsetzung} \\
\hline
Root-Besitz verhindert Privilege Escalation & Alle kritischen Dateien geh\"oren root \\
Gruppenbasierte Zugriffskontrolle & tracegroup f\"ur lesenden Zugriff \\
Principle of Least Privilege & Minimale Rechte f\"ur jeden Prozess \\
\hline
\end{tabular}
}
\end{center}

\vfill
\newpage

% Seite 2
\section*{\textbf{\huge OS-Sicherheit mit authorized\_keys}}
\vspace{-0.3cm}
\noindent\rule{\textwidth}{0.4pt}
\vspace{0.3cm}

\begin{lstlisting}[language=bash,title=authorized\_keys Konfiguration]
# NUR Ansible Controller (10.1.2.3) ist erlaubt!
restrict,no-port-forwarding,no-X11-forwarding,no-agent-forwarding,
no-pty,from="10.1.2.3",command="/usr/local/bin/tracersh" <public-key>
\end{lstlisting}

\textbf{\large OpenSSH Sicherheitsebenen:}

\begin{center}
{\large
\begin{tabular}{|l|l|l|}
\hline
\textbf{Option} & \textbf{Funktion} & \textbf{Sicherheitsvorteil} \\
\hline
restrict & Master-Schalter (ab OpenSSH 7.2) & Deaktiviert ALLE gef\"ahrlichen Features \\
from="10.1.2.3" & IP-Whitelist & \textbf{Exakte IP-Pr\"ufung, keine Netze!} \\
command="..." & Forced Command & \"Uberschreibt jeden Client-Befehl \\
no-pty & Kein Pseudo-Terminal & Verhindert interaktive Sessions \\
no-port-forwarding & Kein TCP/UDP Forwarding & Verhindert Tunnel/Pivoting \\
no-X11-forwarding & Kein X11 & Keine GUI-Weiterleitung \\
no-agent-forwarding & Kein SSH-Agent & Verhindert Key-Weitergabe \\
\hline
\end{tabular}
}
\end{center}

\textbf{\large Dateisystem-H\"artung:}

\begin{center}
{\large
\begin{tabular}{|l|l|l|l|}
\hline
\textbf{Pfad} & \textbf{Besitzer:Gruppe} & \textbf{Rechte} & \textbf{Sicherheitszweck} \\
\hline
/home/traceuser & root:tracegroup & 750 & User kann Home nicht \"andern \\
\textasciitilde/.ssh/ & root:tracegroup & 750 & SSH-Config unver\"anderlich \\
\textasciitilde/.ssh/authorized\_keys & root:tracegroup & 640 & Nur lesbar via Gruppe \\
/usr/local/bin/tracersh & root:root & 755 & Systemweit, unver\"anderlich \\
/var/log/tracersh.log & root:tracegroup & 660 & Audit-Log (optional) \\
\hline
\end{tabular}
}
\end{center}

\vfill
\newpage

% Seite 3
\section*{\textbf{\huge Bash-Script Sicherheitsimplementierung}}
\vspace{-0.3cm}
\noindent\rule{\textwidth}{0.4pt}
\vspace{0.3cm}

\textbf{\large Vollst\"andige Sicherheitsanforderungen im tracersh Script:}

\begin{center}
{\large
\begin{tabular}{|l|l|c|}
\hline
\textbf{Anforderung} & \textbf{Implementierung} & \textbf{Zeile} \\
\hline
Umgebungss\"auberung & LC\_ALL=C LANG=C & 16 \\
Sicheres IFS & IFS=\$'\textbackslash n\textbackslash t' & 17 \\
Fester PATH & PATH='/usr/sbin:/usr/bin:/sbin:/bin' & 18 \\
Variablen-Bereinigung & unset BASH\_ENV ENV CDPATH & 19 \\
Alias-Entfernung & unalias -a & 20 \\
Strict Mode & set -euo pipefail & 39 \\
Input-Validierung & Regex \textasciicircum10\textbackslash.\textbackslash d+\textbackslash.\textbackslash d+\textbackslash.\textbackslash d+\$ & 354 \\
Command Injection Schutz & awk '\{print \$1\}' Extraktion & 344 \\
Timeout-Schutz & timeout --kill-after=5s 60s & 375 \\
Signal-Behandlung & trap f\"ur HUP, INT, TERM & 391 \\
Read-only Variablen & readonly Deklarationen & 42-98 \\
\hline
\end{tabular}
}
\end{center}

\textbf{\large Zugriffskontroll-Logik:}
\begin{itemize}
    \item Keine SSH\_CLIENT → "Nur SSH erlaubt"
    \item SSH mit TTY ohne Befehl → Verweigert
    \item SSH ohne ORIGINAL\_COMMAND → Verweigert
    \item Ung\"ultige IP → "\# input invalid: <ip>"
    \item Nur erstes Wort wird verarbeitet
    \item Keine Shell-Expansion m\"oglich
\end{itemize}

\vfill
\newpage

% Seite 4
\section*{\textbf{\huge Sicherheitsarchitektur und Verbindungsablauf}}
\vspace{-0.3cm}
\noindent\rule{\textwidth}{0.4pt}
\vspace{0.3cm}

\textbf{\large Kompletter Sicherheitsablauf:}

\begin{center}
{\large
\begin{tabular}{|c|l|l|l|}
\hline
\textbf{Schritt} & \textbf{Komponente} & \textbf{Pr\"ufung} & \textbf{Aktion bei Fehler} \\
\hline
1 & Ansible Controller & Initiiert SSH-Verbindung & - \\
2 & OpenSSH Server & from="10.1.2.3" pr\"ufen & Connection refused \\
3 & OpenSSH Server & Public-Key Auth & Permission denied \\
4 & OpenSSH Server & restrict + no-* Optionen & Features deaktiviert \\
5 & OpenSSH Server & command="/usr/local/bin/tracersh" & Forced execution \\
6 & tracersh Script & SSH\_CLIENT vorhanden? & "Nur SSH erlaubt" \\
7 & tracersh Script & SSH\_ORIGINAL\_COMMAND? & "Nur traceroute erlaubt" \\
8 & tracersh Script & Target = 10.x.x.x? & "\# input invalid: <ip>" \\
9 & tracersh Script & Tool verf\"ugbar? & Fallback oder Error \\
10 & System & traceroute/mtr mit Timeout & Timeout nach 60s \\
11 & tracersh Script & Parse zu CSV & Error message \\
12 & OpenSSH Server & Output an Client & Connection closed \\
\hline
\end{tabular}
}
\end{center}

\textbf{\large Testabdeckung:} ansible-playbook -i "host," test.yml

\begin{center}
\begin{tabular}{|l|l|l|}
\hline
\textbf{Testbereich} & \textbf{Abdeckung} & \textbf{Pr\"ufung} \\
\hline
Konnektivit\"at & SSH-Verbindung, Schl\"ussel-Auth & Nur 10.1.2.3 kann sich verbinden \\
Befehlsausf\"uhrung & G\"ultige/ung\"ultige Targets & Nur 10.x.x.x IPs werden akzeptiert \\
Sicherheitsrestriktionen & Verbotene Befehle, Shell-Zugriff & Keine interaktive Shell m\"oglich \\
Fehlerbehandlung & Timeouts, fehlende Tools & 60s Timeout, Fallback zu mtr \\
Idempotenz & Mehrfache Ausf\"uhrungen & Keine Seiteneffekte bei Wiederholung \\
\hline
\end{tabular}
\end{center}

\end{document}